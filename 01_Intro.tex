\section{Introduction}

Legal applications have often been used as test cases for knowledge-based AI systems (e.g., \cite{cacm/SergotSKKHC86}). %, Zurek12}. %This was the case for traditional rule-based expert systems \cite{cacm/SergotSKKHC86}, but also for more recent systems such as, DROOLS \cite{Zurek12}. 
In \cite{ruleml/DeryckHVV18}, an interactive decision enactment system for notaries was developed according to the Knowledge Base Paradigm (KBP) \cite{iclp/DeneckerV08}. 

The KBP advocates a strict separation between declarative domain knowledge and the use of this knowledge to perform certain tasks. 
This separation allows the same knowledge to be used by different \emph{inference} algorithms in order to achieve different goals.
\begin{comment}
This is in contrast to  typical rule-based expert systems in which knowledge is formalized specifically with a forward-chaining inference algorithm in mind, or Prolog-based systems in which knowledge is formalized with a backwards query-answering algorithm in mind. 

In the KBP, domain knowledge is formalized as a purely declarative \emph{knowledge base}, which is not tied to a specific \emph{inference} method, allowing the same knowledge to be reused by different algorithms in order to achieve different goals.
\end{comment}
As claimed by \cite{iclp/DeneckerV08}, this paradigm has two main advantages. First, the knowledge base is easier to maintain, because it can be considered in isolation from the inference methods.  Second, the knowledge base is easier to reuse for other inference tasks, since it is not tied to any specific inference method anyway.

In \cite{ruleml/DeryckHVV18}, a decision enactment system that supports Belgian notaries in their handling of real estate sales was developed according to the KBP. The Belgian legislation on registration duties that need to be paid when purchasing real estate is quite complex: there exist multiple registration types with different rates, and legislation from the country's three regions may apply in addition to federal regulations. The tool was developed together with notary Luc Van Pelt. Like other Belgian notaries, his office prides itself on its customer-friendly and confidential service. Therefore, he is looking for a system that provides support while interviewing clients, without interrupting the natural flow of the conversation.% The system should therefore be able to accept relevant information in any order and to provide useful feedback  on each piece of information.

In this paper, we further analyze and develop the prototype that was developed in \cite{ruleml/DeryckHVV18}. This work focuses on validating the two advantages of the KBP mentioned above.
First, we update the prototype to cope with a recent change to Belgian legislation. This change was significant enough to warrant substantial coverage by major Belgian news outlets and therefore presents an interesting and representative test case for the maintainability of the knowledge base.
Second, during its evaluation of the prototype, the notary office identified additional desirable features that were not initially thought of. 
We were able to add these features to the prototype in a generic, domain-independent way. This supports the claim that the functionality that users desire can indeed be implemented by applying domain-independent inference methods to a purely declarative knowledge base, even if this functionality was not originally foreseen when the knowledge base was constructed.

This work results in a completely generic framework, similar to, but more powerful than that of \cite{ruleml/DassevilleJJVD16}. This generic framework can be applied to create powerful interactive decision enactment systems for other domains with minimal effort. It is developed using the \idp KBP system \mycite{IDP}, which allows it to benefit from both this system's expressive knowledge representation language \fodot, as well as from its efficient inference algorithms.
 
The next section elaborates on the background of the case, the main characteristics of the original prototype and the changes in legislation.
Section \ref{KBP} introduces the KBP and the \idp system.
%In Section \ref{original} the main characteristics of the original prototype are brought to mind.
Section \ref{interface} elaborates on new inferences in the revised interface.  
Section \ref{related} presents related work, followed by a discussion and conclusion in Section \ref{conclusion}.