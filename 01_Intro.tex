\section{Introduction}

As most legislation, the Belgian legislation on registration duties with the purchase of real estate is complex.
Although a simplification of the applicable rules took place in 2018, there still exist multiple registration types that differ in the different communities of the country.
Moreover, as the legislation changed recently, no decent online support for interested buyers is available.
%%For citizens it was difficult to know the amount of duties to be paid. 
%%They would need to seek the advice of experts.
%%This could often give an indication of the due amount.
%%owever, to be sure the notary would need to perform a thorough investigation, hence taking up precious time in the bargaining process when purchasing a house.
Because of the amplitude of the legislation, the multiplex exceptions and the variety of definitions attributed to some concepts, the risks of mistakes and omissions are numerous, also for experts like notaries.
%%Even after the simplification of the applicable law in 2018, this remains the case.

At the same time, many notaries still seem to be working in the old school way, with static printed questionnaires to gather information.
The retrieval of the information requested in this questionnaire is sometimes cumbersome, and might include information like the contract date of previous purchases or sells, or the ownership share of previous owned real estate.
Both the risk of omissions and the burdensome retrieval of information lead to the current project, in which an application was developed to support the notary in the decision making and in the relevant information detection.
The case emerged at the request of notary Van Pelt, but is in the first place intended for general use by Belgian notaries.
Evidently, also the public administration that bills the registration duty could use the application.
Finally, due to its simplicity and included explanation features, also candidate home owners could use it to get a solid idea of the payable duties.

We developed an application with IDP.  
This is a knowledge based system (KBS) that uses a derivative of First Order Logic to create a knowledge base that can be used in multiple ways with several forms of inference.
Thanks to these different inferences, consequences of choices are be explicitated, superfluous questions can be skipped and the final duty to be applied is calculated.

...
